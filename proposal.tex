\documentclass[12pt,conference,onecolumn]{IEEEtran}
\usepackage{graphicx}
\usepackage{amsmath}
\usepackage{url}
\usepackage{mathrsfs}
\usepackage{float}

%\usepackage[citestyle=ieee,backend=biber]{biblatex}
%\addbibresource{bibliography.bib}

% \usepackage[ruled]{algorithm2e}
% \SetKwProg{Fn}{Function}{:}{}
% \SetAlgoFuncName{Ftn.}{Function}
% \SetAlgorithmName{Alg.}{Algorithm}{List of Algorithms}

\begin{document}

\title{Proposal}
\author{
	\IEEEauthorblockN{Stephen Brett}
	\IEEEauthorblockA{The Cooper Union\\New York, NY, USA\\Email: brett@cooper.edu}
	\and
	\IEEEauthorblockN{Daniel Nakhimovich}
	\IEEEauthorblockA{The Cooper Union\\New York, NY, USA\\Email: nakhimov@cooper.edu}
	\and
	\IEEEauthorblockN{Ostap Voynarovskiy}
	\IEEEauthorblockA{The Cooper Union\\New York, NY, USA\\Email: nakhimov@cooper.edu}
}

\maketitle
\renewcommand{\baselinestretch}{1.5} 
\begin{abstract}
\textbf{\\Project dates: September 2018 -- December 2018\\}
\textnormal{
	Abstract
}
\end{abstract}
%\onecolumn \maketitle \normalsize \vfill
\IEEEpeerreviewmaketitle

\section{Project Description} \label{sec:proj_desc}

\section{Background} \label{sec:background}
\subsection{Eye Tracking}
Eye gaze tracking (EGT) systems are used to estimate the location of the point at which a person is looking. There are both intrusive and non-intrusive methods of taking the necessary measurements. Intrusive methods involve physical contact of the tracking system with the user
. One such method uses sceleral search coils, which are contact lenses that have been modified to have a coil of wire embedded in them. In a magnetic field, the position of the coil, and thus the direction in which the eye is looking, can easily and quickly be determined. This method is generally considered to be the most accurate, and is often used in medical research \cite{chennamma2013survey}. Yet since this method is invasive, requiring the insertion of a device into the body, it is not practical for commercial use. An example of intrusive tracking that is not invasive is electro-oculography, in which sensors, are adhered to the skin in the area around the eyes, and detect differences in electric potential caused by eye rotation. This is much less expensive, but is also less accurate, only to $2^{\circ}$ \cite{morimoto}.

A method of eye tracking that is conducive to everyday use involves video processing of the user's face and eyes. This can be either intrusive or non-intrusive, or even both, depending on where the cameras are mounted. For an intrusive approach, the user would wear a headset with one or more cameras attached, possibly along with light sources. With a non-intrusive method, the camera would be mounted remote from the user, giving them more freedom of motion. For this project, we are going to focus on video-based eye gaze tracking.

\subsection{Video-Based Eye Tracking}

\section{Proposition} \label{sec:proposition}

\bibliographystyle{IEEEtran}
\bibliography{bibliography}
%\printbibliography

\end{document}
